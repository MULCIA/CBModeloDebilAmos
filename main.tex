\documentclass[12pt]{article}
% pre\'ambulo
\usepackage{lmodern}
\usepackage[T1]{fontenc}
\usepackage[spanish,activeacute]{babel}
\usepackage{mathtools}
\usepackage{url}
\usepackage{tikz}
\usepackage{graphicx}
\graphicspath{ {images/} }
\usetikzlibrary{arrows,positioning}
\tikzset{
    %Define standard arrow tip
    >=stealth',
    %Define style for boxes
    punkt/.style={
           rectangle,
           rounded corners,
           draw=black, very thick,
           text width=5em,
           minimum height=4em,
           text centered},
    % Define arrow style
    pil/.style={
           ->,
           thick,
           shorten <=2pt,
           shorten >=2pt,}
}
\usepackage[style=numeric-comp,backend=bibtex]{biblatex}
\bibliography{refs.bib}

\title{Dise\~no y verificaci\'on formal de programas moleculares en el modelo d\'ebil de Amos}
\author{Sergio Rodr\'iguez Calvo}
\date{Septiembre de 2017}


\begin{document}
    \maketitle
    \thispagestyle{empty}
    \begin{center}
      Departamento de Ciencias de la Computaci\'on e
      Inteligencia Artificial \\
      Universidad de Sevilla
      \end{center}
  % cuerpo del documento
  % abstract
  \bf{Abstract. }\rm
    \emph{En el presente trabajo se pretende estudiar el dise'no y la verificaci'on formal
    de dos programas moleculares en el modelo d'ebil de Amos que resuelven el problema de la generaci'on
    de permutaciones; y el problema del camino hamiltoniano en su versi'on dirigida y sin
    nodos distinguidos. Consiste en un trabajo desarrollado como trabajo final de la asignatura de
    Computaci'on Bioinspirada y consiste en la reescritura de un apartado }

\section{Introducci'on}

En la d'ecada de los cincuenta comienza a ser evidente que existe una analog'ia entre algunos procesos
matem'aticos y ciertos procesos biol'ogicos. Un organismo vivo puede ser visto como el resultado de
aplicar una serie de operaciones bioqu'imicas sobre una cadena de 'acido desoxirribonucleico (ADN).

Posteriormente, en la decada de los noventa, se demostr'o que se pueden usar ciertos procesos biol'ogicos
para atacar la resolubilidad de problemas matem'aticos dif'iciles. Estos problemas tambi'en son conocidos
como computacionalmente intratables, y son aquellos que su soluci'on algor'itmica toma una cantidad de
recursos exponenciales en el tama'no del dato de entrada.

Esta resolubilidad est'a relacionada con la potencia de c'alculo y la densisdad de almacenamiento de los
ordenadores convencionales.

En la propia d'ecada de los cincuenta ya se introdujo el concepto te'orico de computaci'on a nivel molecular.
En los ordenadores convencionales la paralelizaci'on y la miniturizaci'on son un objetivo importante, y la
computaci'on molecular puede suponer un paso m'as en este sentido.

Sobre todo, a partir de que en la d'ecada de los ochenta se demostrara que existia un limite en la potencia
de c'alculo y en la miniturizaci'on de los componentes electr'onicos empleados en los ordenadores convencionales.

La computaci'on molecular cuenta con las siguientes ventajas:

\begin{itemize}
	\item Sustituci'on de la luz por reacciones qu'imicas, lo que implica un ahorro del consumo energ'etico.
	\item El uso de interruptores moleculares permite, seg'un se estima, disponer de m'as de mil procesadores
    en el mismo espacio que un procesador convencional.
	\item Se estima que los interruptores moleculares pueden aumentar cien mil millones de veces la capacidad
    de procesamiento respecto a los ordenadores convencionales.
	\item Se estima que se podr'ian reproducir la capacidad de cien ordenadores en el tama'no de un grano de sal fina.
\end{itemize}

\section{Modelo d'ebil de Amos}

El Modelo d'ebil de Amos es uno de los modelos de computaci'on molecular sin memoria basados en ADN.
Previamente, se describe brevemente el modelo abstracto de este tipo de computaci'on para entender mejor
el modelo que se introducce en esta secci'on.

Posteriormente, se resolveran dos problemas considerados dif'iciles, que son el problema de la generaci'on de
permutaciones, as'i como, el problema del camino hamiltoniano, en concreto en su versi'on dirigida
y sin nodos distinguidos.

\subsection{Modelo abstracto de computaci'on molecular}

La computaci'on molecular que se presenta es la que se realiza a trav'es del ADN. Por ello, se presenta una
introducci'on breve para llegar a explicar completamente el modelo abstracto de computaci'on molecular.

\subsubsection{Estructura del ADN}

En primer lugar, el ADN es un polimero con estructura lineal formado por nucle'otidos. A su vez cada
nucle'otido consta de:

\begin{itemize}
	\item Base nitrogenada, unido a un azucar a trav'es del carbono 1.
	\item Un azucar de cinco 'atomos de carbono.
	\item Un grupo hidroxilo (OH) unido al azucar por el carbono 3.
	\item Un grupo fosfato (P) unido al azucar por el cabrono 5.
\end{itemize}

#TODO: Poner aqu'i imagen de la p'agina 6.

El nucleotido se identifica con la base nitrogenada, existiendo cuatro tipos: adenina (A), citosina (C),
guanina (G) y timina (T). La adenina y la guanina pertenecen al grupo de las purinas, y las otras dos
al grupo de las piramidas.

Estos nucleotidos pueden enlazarse a trav'es de dos tipos de enlaces, enlace fosfodieste, a trav'es
la uni'on entre el grupo fosfato de un nucleotido con un grupo hidroxilo de otro. O, enlace de hidrogeno,
que se realiza a trav'es de las bases.

El enlace fosfoide permite crear cadenas simples que poseen dos extremos de comportamientos diferentes. El grupo
de los extremos que queda disponible para continuar con nuevos enlaces dan una polaridad. Por tanto, tenemos dos
polaridades diferentes. Direcci'on 5´ -> 3´ y 3´-> 5´. Por defecto, si no se indica nada, se trabaja con 5´ -> 3´
y se notar'a #TODO poner aqu'i s'imbolo.

El enlace de hidrogeno se rige por el principio de complementariedad, esto es, la adenina s'olo se
puede unir con la timina, y viceversa, con s'olo dos puentes de hidr'ogeno. La citosina con la guanina,
 y viceversa, con tres puentes de hidr'ogeno.

 Combinando ambos tipos de enlaces se obtiene cadenas dobles, tambi'en llamado doble h'elice. Las dos cadenas
 se alinean de forma antiparalela, esto es, una en direcci'on 5´ -> 3´ y la otra en 3´-> 5´.

#TODO: Poner aqu'i imagen de la p'agina 8 con la d'oble h'elice.

\subsubsection{Operaciones con cadenas de ADN}

A continuaci'on se describe un ejemplo de operaci'on con cadenas de ADN con detalle, y se indicar'a s'olamente
el resto como breve descripci'on.



\subsection{Modelo de Amos}

\section{Problema de la generaci'on de permutaciones}

\section{Problema del camino hamiltoniano en versi'on dirigida sin nodos distinguidos}

% bibliography
\printbibliography

\end{document}
