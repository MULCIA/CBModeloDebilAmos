\documentclass[12pt]{article}
% pre\'ambulo
\usepackage{lmodern}
\usepackage[T1]{fontenc}
\usepackage[spanish,activeacute]{babel}
\usepackage{mathtools}
\usepackage{url}
\usepackage{tikz}
\usepackage{graphicx}
\graphicspath{ {images/} }
\usetikzlibrary{arrows,positioning}
\tikzset{
    %Define standard arrow tip
    >=stealth',
    %Define style for boxes
    punkt/.style={
           rectangle,
           rounded corners,
           draw=black, very thick,
           text width=5em,
           minimum height=4em,
           text centered},
    % Define arrow style
    pil/.style={
           ->,
           thick,
           shorten <=2pt,
           shorten >=2pt,}
}
\usepackage[style=numeric-comp,backend=bibtex]{biblatex}
\bibliography{refs.bib}

\title{Dise\~no y verificaci\'on formal de programas moleculares en el modelo d\'ebil de Amos}
\author{Sergio Rodr\'iguez Calvo}
\date{Septiembre de 2017}


\begin{document}
    \maketitle
    \thispagestyle{empty}
    \begin{center}
      Departamento de Ciencias de la Computaci\'on e
      Inteligencia Artificial \\
      Universidad de Sevilla
      \end{center}
  % cuerpo del documento
  % abstract
  \bf{Abstract. }\rm
    \emph{En el presente trabajo se pretende estudiar el dise'no y la verificaci'on formal
    de dos programas moleculares en el modelo d'ebil de Amos que resuelven el problema de la generaci'on
    de permutaciones; y el problema del camino hamiltoniano en su versi'on dirigida y sin
    nodos distinguidos. En concreto, se centra en reescribir, as'i como, completar algunos detalles
    de las demostraciones, que se encuentran el trabajo original \cite{Mario-deJesus}.
    El desarrollo de este trabajo tiene como objetivo ser entregado como trabajo final de la asignatura de
    Computaci'on Bioinspirada.}

\section{Introducci'on}

En la d'ecada de los cincuenta comienza a ser evidente que existe una analog'ia entre algunos procesos
matem'aticos y ciertos procesos biol'ogicos. Un organismo vivo puede ser visto como el resultado de
aplicar una serie de operaciones bioqu'imicas sobre una cadena de 'acido desoxirribonucleico (ADN).

Posteriormente, en la decada de los noventa, se demostr'o que se pueden usar ciertos procesos biol'ogicos
para atacar la resolubilidad de problemas matem'aticos dif'iciles. Estos problemas tambi'en son conocidos
como computacionalmente intratables, y son aquellos que su soluci'on algor'itmica toma una cantidad de
recursos exponenciales en el tama'no del dato de entrada.

Esta resolubilidad est'a relacionada con la potencia de c'alculo y la densisdad de almacenamiento de los
ordenadores convencionales.

En la propia d'ecada de los cincuenta ya se introdujo el concepto te'orico de computaci'on a nivel molecular.
En los ordenadores convencionales la paralelizaci'on y la miniturizaci'on son un objetivo importante, y la
computaci'on molecular puede suponer un paso m'as en este sentido.

Sobre todo, a partir de que en la d'ecada de los ochenta cuando se demostr'o la existencia un l'imite en la potencia
de c'alculo y en la miniturizaci'on de los componentes electr'onicos empleados en los ordenadores convencionales.

Por 'ultimo se enumeran las principales ventajas del uso de la computaci'on molecular:

\begin{itemize}
	\item Sustituci'on de la luz por reacciones qu'imicas, lo que implica un ahorro del consumo energ'etico.
	\item El uso de interruptores moleculares permite, seg'un se estima, disponer de m'as de mil procesadores
    en el mismo espacio que un procesador convencional.
	\item Se estima que los interruptores moleculares pueden aumentar cien mil millones de veces la capacidad
    de procesamiento respecto a los ordenadores convencionales.
	\item Se estima que se podr'ian reproducir la capacidad de cien ordenadores en el tama'no de un grano de sal fina.
\end{itemize}

\section{Modelo d'ebil de Amos}

Antes de introducir el Modelo d'ebil de Amos, se necesita previamente conocer los detalles y principios
de la computaci'on molecular, as'i como, los distintos modelos previos a este, los cuales se pueden encontrar
en el documento original del profesor del departamento de Ciencias de la Computaci'on de la Universidad
de Sevilla, Mario de Jes'us P'erez Jim'enez \cite{Mario-deJesus}.

El Modelo d'ebil de Amos consiste en modelo de computaci'on basada en ADN, esto es, que utiliza como sustrato
computacional el ADN y en el cual se realizan filtrados sobre el sustrato anterior. En este caso, no existe
memoria de acceso aleatorio como en la computaci'on cl'asica. Para almacenar el sustrato, al igual que en otros
modelos de computaci'on molecular, se utiliza un tubo de ensayo que contendr'a la muestra.

Dicho tubo es un multiconjunto finito de cadenas del alfabeto $\sum_{ADN} = \{A, C, G, T\}$.

A nivel abstracto, las operaciones que se pueden realizar sobre los tubos son las siguientes:

\begin{itemize}
    \item Quitar ...
    \item Copiar ...
    \item Uni'on ...
    \item Selecci'on ...
\end{itemize}

Estas operaciones ser'an instrucciones moleculares primitivas del modelo d'ebil.

El primer problema, el de la generaci'on de permitaciones, ser'a abordado previo al problema del camino hamiltoniano
, ya que, ser'a necesario para su resoluci'on.

\subsection{Problema de la generaci'on de permutaciones}

Una permutaci'on la definimos como dado un numero natural, $n \geq 1$, una permutaci'on de orden n es una
aplicaci'on biyectiva del conjunto finito \{1,...,n\} en s'i mismo.

Una vez tenemos definido qu'e es una permutaci'on vamos a introducir el problema de generaci'on de
permutaciones: Dado un n'umero natural, n >= 2, generar todas las permutaciones de orden n.

\subsection{Problema del camino hamiltoniano en versi'on dirigida sin nodos distinguidos}

El problema del camino hamiltoniano en su versi'on dirigida y sin nodos distinguidos consiste en
dado un grafo dirigido, determinar si existe un camino simple que pasa por todos los nodos del grafo. O lo
que es lo mismo, si el grafo posee un ciclo hamiltoniano.

% bibliography
\printbibliography

\end{document}
