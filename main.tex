\documentclass[12pt]{article}
% pre\'ambulo
\usepackage{lmodern}
\usepackage[T1]{fontenc}
\usepackage[spanish,activeacute]{babel}
\usepackage{mathtools}
\usepackage{url}
\usepackage{tikz}
\usepackage{graphicx}
\graphicspath{ {images/} }
\usetikzlibrary{arrows,positioning}
\tikzset{
    %Define standard arrow tip
    >=stealth',
    %Define style for boxes
    punkt/.style={
           rectangle,
           rounded corners,
           draw=black, very thick,
           text width=5em,
           minimum height=4em,
           text centered},
    % Define arrow style
    pil/.style={
           ->,
           thick,
           shorten <=2pt,
           shorten >=2pt,}
}
\usepackage[style=numeric-comp,backend=bibtex]{biblatex}
\bibliography{refs.bib}

\title{Dise'no y verificaci'on formal de programas moleculares en el modelo d'ebil de Amos}
\author{Sergio Rodr\'iguez Calvo}
\date{Septiembre de 2017}


\begin{document}
    \maketitle
    \thispagestyle{empty}
    \begin{center}
      Departamento de Ciencias de la Computaci\'on e
      Inteligencia Artificial \\
      Universidad de Sevilla
      \end{center}
  % cuerpo del documento
  % abstract
  \bf{Abstract. }\rm
    \emph{En el presente trabajo se pretende estudiar el dise'no y la verificaci'on formal
    de dos programas moleculares en el modelo d'ebil de Amos que resuelven el problema de la generaci'on
    de permutaciones; y el problema del camino hamiltoniano en su versi'on dirigida y sin
    nodos distinguidos.}

\section{Introducci'on}

Hace algunas d'ecadas, los investigadores comenzaron a ver la analog'ia existente entre algunos
procedimientos matem'aticos y ciertos procesos biol'ogicos. Es decir, las c'elulas de los organismos
vivos, aplicando una serie de operaciones bioqu'imicas sobre una cadena de 'acido desoxirribonucleico (ADN),
realizan operaciones que son computables, tales como, la suma o la resta.

A ra'iz de el hallazgo de los l'imites en la miniaturalizaci'on de los componentes de tecnolog'ia electr'onica (chips)
empleados en los ordenadores modernos, surgen estudios y avances en cuanto a utilizar computaci'on
molecular como alternativa, ya que todo parece indicar que con suficientes avances en este campo
se puede aprovechar la miniaturizaci'on obtenida por procesos biol'ogicos y evolutivos, as'i como, el
menor consumo energ'etico de una c'elula frente a un transistos electr'onico.

Aunque en la actualidad a'un no se ha conseguido que sean pr'acticos, parece existir cierta esperanza y
cierta posibilidad, de que con el suficiente avance se pueda utilizar de manera pr'actica. De este modo,
obtener resultado en un tiempo menor, utilizando para ello espacio exponencial.

De hecho, en la d'ecada de los noventa se mostr'o que era posible atacar la resolubilidad de un problemas
matem'aticos dif'iciles, es decir, aquellos problemas que necesitan una cantidad exponencial de recursos
en el tama'no de dato de entrada.

\section{Modelo d'ebil de Amos}

\section{Problema de la generaci'on de permutaciones}

\section{Problema del camino hamiltoniano en versi'on dirigida sin nodos distinguidos}

% bibliography
\printbibliography

\end{document}
